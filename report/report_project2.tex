\documentclass[10pt,conference,compsocconf]{IEEEtran}

\usepackage{hyperref}
\usepackage{graphicx}	% For figure environment
\usepackage{amsmath}
\usepackage{amssymb}
\usepackage{float}

\newcommand{\beginsupplement}{%
	\setcounter{table}{0}
	\renewcommand{\thetable}{S\arabic{table}}%
	\setcounter{figure}{0}
	\renewcommand{\thefigure}{S\arabic{figure}}%
}

\begin{document}
\title{CS-433 Machine Learning Project 1}

\author{
  Matthias Minder, Zora Oswald, Silvan Stettler\\
}

\maketitle

\begin{abstract}
Abstract...
\end{abstract}

\section*{Introduction} 
In recent years, the emergence of single cell RNA sequencing (scRNA-seq) has lead to the discovery of many new cell types based on their gene expression profile. The question naturally arises whether it would be possible to predict cell types using a machine learning approach. Of special interest is the detection and \textit{de novo} discovery of stem cells in tissues for which no stem cell population has been characterized. This project aims to create a classifier able to assign a "stemness" score to cells based on scRNA-seq data. 
\par
However, several challenges are associated with working with scRNA-seq data: 
\section*{Methods}

\section*{Results}

\section*{Conclusion}

%%% Bibliography
%\bibliographystyle{IEEEtran}
%\bibliography{literature-project2}


\end{document}
